\documentclass[12pt]{article}
\usepackage{amssymb}
\usepackage{enumerate}

\oddsidemargin  0.0in
\evensidemargin 0.0in
\textwidth      6.5in
\headheight     0.0in
\topmargin      -0.2in
\textheight     8.7in






\begin{document}
\begin{center}
{\bf Jack H. Lutz - Short CV}
\end{center}




\noindent{\bf Current position:}
Professor of Computer Science, Iowa State University  \\
\indent (Assistant Professor 1987-1992; Associate Professor 1992-1996; Professor 1996-present)
\indent Professor of Mathematics \\
\indent Faculty Member in Bioinformatics and Computational Biology 

\vspace{0.1cm}


\noindent{\bf Visiting positions}

\indent
Visiting Associate, Caltech, 2020

\indent
Visitor, University of Oxford, 2017

\indent
Visitor, University of Wisconsin, 2017

\indent
Visiting Associate, Caltech, 2017

\indent
Visiting Fellow, University of Cambridge, 2012

\indent
Visiting Associate, Caltech, 2012

\indent
Visiting Professor, University of Wisconsin, 2006


\indent Visiting Scientist, NEC Research Institute, 2001

\indent
Visiting Professor, Cornell University, 1997

\indent
DIMACS Visiting Fellow, Rutgers University, 1990


\vspace{0.2cm}

\noindent{\bf Research areas}

\indent
Molecular Programming and DNA Nanotechnology: universality,
robustness, dynamics, specification, and verification of
programmable self-assembling nanosystems.

\indent
Algorithmic Information and Randomness: algorithmic
dimensions, Kolmogorov complexity, randomness, prediction,
finite-state dimension, and algorithmic fractal geometry.

\indent
Computational Complexity: complexity in analysis, structure of 
complexity classes, and resource-bounded measure and dimension.


\vspace{0.2cm}
\noindent{\bf Research funding} (Principal Investigator unless noted)

\begin{tabular}{@{\hspace*{-2pt}}l@{\hspace*{43pt}}l}
NSF Grant, 2019--2023 (coPI) & NSF Grant, 1997--2000 \\
NSF Grant, 2015--2021 (coPI) & NSF Presidential Young Investigator Award, 1991--1997  \\   
NSF Grant, 2012--2017 & Rockwell International, 1991--1996  \\
NSF Grant, 2011--2013 & Amoco Foundation, 1993--1995  \\
NSF Grant, 2007--2012 & Microware Systems Corporation, 1992--1996  \\
NSF Grant, 2007--2010 & Hughes Aircraft Company 1990--1991  \\
NSF Grant, 2003--2006 & NSF Grant, 1988--1991 \\
NSF Grant, 2000--2004 &
\end{tabular}


\vspace{0.2cm}

\noindent{\bf Publications} (authored and co-authored)

\indent
60 journal papers, principally in {\it SIAM Journal on Computing},
{\it Information and Computation},
{\it Theoretical Computer Science},
{\it Journal of Computer and System Sciences},
and {\it Theory of Computing Systems}

\indent
68 conference papers


\vspace{0.2cm}


\noindent{\bf Recent and upcoming invited lectures at meetings}

\indent Conference on Computability, Complexity, and Randomness, 2022

\indent Minisymposium on Applications of Stochastic Reaction Networks, SIAM-DS 2021

\indent Algorithmic Randomness Workshop, American Institute of Mathematics, 2020

\indent AMS-ASL Special Session: Logic Facing Outward, Joint Mathematics Meetings, 2020

\indent ASL North American Annual Meeting, New York, 2019 (two-hour tutorial)

\indent Equidistribution: Arithmetic, Computational and Probabilistic Aspects, NUS IMS, 2019

\indent Computability Workshop, Oberwolfach Mathematics Research Institute, 2018

\indent Midwest Computability Seminar, Chicago, 2017

\indent NZMRI Workshop and Summer School, Napier, NZ, 2017 (three lectures) 

\indent Workshop on Normal Numbers, Erwin Schr\"{o}dinger Institute, 2016

\indent AMS Special Session on Effective Mathematics in Discrete and Continuous Worlds, 2016

\indent Conference on Computability, Complexity, and Randomness (CCR), 2015

\indent Computability Special Session, ASL North American Annual Meeting, 2015

\indent Midwest Computability Seminar, Chicago, 2014

\indent AMS-ASL Special Session on Logic and Probability, Joint Math Meetings, 2014

\indent Natural Algorithms and the Sciences Workshop, Princeton, 2013

\indent Intl. Conference on Unconventional Computation and Natural Computing (UCNC), 2012

\indent Conference on Computability, Complexity, and Randomness (CCR), 2012

\indent Logic, Dynamics, and their Interactions (celebrating work of Dan Mauldin), 2012

\indent Intl. Conference on Languages and Automata Theory and Applications (LAA), 2012

\indent Conference on DNA Computing and Molecular Programming (DNA), 2011

\indent Conference on Computability in Europe (CiE), 2011 (three-hour tutorial)

\indent AMS-ASL Special Session on Logic and Analysis, Joint Math Meetings, 2011

\indent Conference on Computability, Complexity, and Randomness (CCR), 2010

\indent CiE Special Session on Algorithmic Randomness, 2009

\indent Workshop on Fractals and Tilings, 2009

\indent Conference on Logic, Computability, and Randomness, 2009



\vspace{0.2cm}


\noindent{\bf Education}
   
\indent Ph.D., Mathematics, Caltech, 1987 (Adviser: Alexander S. Kechris)

\indent M.S., Computer Science, Univ. Kansas, 1981

\indent M.A., Mathematics, Univ. Kansas, 1979

\indent B.G.S., Mathematics, Univ. Kansas, 1976


\vspace{0.2cm}

\noindent{\bf Ph.D. Degrees Supervised}

\indent Xiang Huang, 2020, now Assistant Professor, Univ. Illinois--Springfield

\indent Donald M. Stull, 2017, now Lecturer, Iowa State University

\indent Adam Case, 2016, now Assistant Professor, Drake Univ.

\indent Titus Klinge (two advisers), 2016, now Assistant Professor, Drake Univ.

\indent Divita Mathur (two advisers), 2016, now Postdoctoral Fellow, U.S. Naval Research Lab

\indent Brian Patterson (two advisers), 2011, now Associate Professor, Oglethorpe Univ.

\indent Scott M. Summers, 2010, now Associate Professor, Univ. Wisconsin--Oshkosh

\indent Matthew J. Patitz, 2010, now Associate Professor, Univ. Arkansas

\indent Xiaoyang Gu, 2009, now Head of Data Engineering, BorderX Lab

\indent Satyadev Nandakumar, 2009, now Associate Professor, IIT--Kanpur, India

\indent David S. Doty (two advisers), 2009, now Assistant Professor, Univ. California--Davis

\indent John M. Hitchcock, 2003, now Professor, Univ. Wyoming

\indent Jack J. Dai (two advisers), 2001, now Postdoctoral Fellow, Fudan University, China

\indent James I. Lathrop, 1997, now Assistant Professor, Iowa State Univ.

\indent Josef M. Breutzmann, 1996, now Professor Emeritus, Wartburg College

\indent David W. Juedes, 1994, now Professor and Chair, Ohio Univ.

\indent External committee member for Ph.D. students at U. Zaragoza (2011), U. Illinois (2005),
U. Amsterdam (1998), Heidelberg U. (1996), Rutgers U. (1995), and Polytech. U. Catalonia (1994)

\indent Currently supervising four Ph.D. students at Iowa State University



\vspace{0.2cm}

\noindent{\bf Conference program committee memberships}

\indent
Intl. Conf. on Computability, Complexity, and Randomness (CCR) 2022, 2016, 2010, 2008

\indent
ACM Intl. Conf. on Nanoscale Computing and Communication (NanoCom) 2016

\indent Conference on Computability in Europe (CiE) 2015

\indent Intl. Conf. on Languages and Automata Theory and Applications (LATA) 2014, 2013

\indent
Intl. Conf. on Computability and Complexity in Analysis (CCA) 2012, 2008, 2005

\indent
Intl. Computing and Combinatorics Conf. (COCOON) 2011

\indent 
IEEE Conference on Computational Complexity (CCC) 2007, 1996, 1993

\indent
Symp. on Theoretical Aspects of Computer Science (STACS) 1998

\end{document}
